\documentclass{prosper}%[pdf,azure,slideColor,colorBG]{prosper}
\hypersetup{pdfpagemode=FullScreen}

\usepackage{graphicx}
\usepackage{graphics}
\usepackage{rotating}
\usepackage{amsmath}
\usepackage{epsfig}
\usepackage{verbatim}

\newcommand{\enfa}{{$\epsilon$-NFA}}
\newcommand{\e} {{\mbox{$\epsilon$}}}
\newcommand{\ra}{\mbox{$\;\rightarrow\;$}}
\newcommand{\vb}{\mbox{$\;|\:$}}
\newcommand{\dra}{\mbox{$\;\Rightarrow\;$}}
\newcommand{\cvd}{\mbox{$\;\overset{*}{\vdash}\;$}}
\newcommand{\cavd}[1]{{\mbox{$\;\overset{_*}{\underset{_#1}{\vdash}}\;$}}}
\newcommand{\avd}[1]{\mbox{$\;{\underset{_#1}{\vdash}\;}$}}
\newcommand{\lmd}{\mbox{$\;\underset{lm}{\Rightarrow}\;$}}
\newcommand{\rmd}{\mbox{$\;\underset{rm}{\Rightarrow}\;$}}\newcommand{\pda}[1]{$#1 = (Q, \Sigma, \Gamma, \delta, q_0, Z_0, F)$}
\newcommand{\qed}{{$\;\;\Box$}}

\newcommand{\quiz}[1]{{\hfill \tiny{$_{ #1}$}}}


%\title{Pushdown Automata}
%\author{Marcus V. dos Santos}
%\institution{        Department of Computer Science \\        Ryerson University}

% Optional: text to put in the bottom of each slide.
% By default, the title of the talk will be placed there.
\slideCaption{\textit{Marcus Santos, RU}}

\begin{document}

\begin{slide}{Procedural descriptors for CFLs}
\begin{itemize}
\item In our study of regular languages  we saw that  we can describe such languages ``procedurally'' using a finite automaton (DFA, NFA, e-NFA), or ``declaratively'' using a regular expression.
\item In our study of context free languages, we first learned how we can use CFGs to ``declaratively'' describe those languages.
\item Today we will learn how we can specify machines called {\blue Pushdown Automata} that describe CFLs ``procedurally''.
\end{itemize}
\end{slide}

%\maketitle
\begin{slide}{Pushdown Automata}
%\begin{center}
%\includegraphics[scale=.3]{figures/pda1.eps}
%\end{center}
\begin{itemize}
\item A pushdown automata (PDA) is essentially an \e-NFA with a stack on which it can store symbols.
\item PDA can only access information on the stack in a last-in-first out way. 
\item There are two versions of  a PDA, and they differ on how a PDA accepts a string, i.e.,
\begin{itemize}
\item acceptance by accepting state, or 
\item acceptance by empty stack.%\item Real computers, 
\end{itemize}
\end{itemize}
\end{slide}

\begin{slide}{PDA informally}
\begin{center}
\includegraphics[scale=.4]{figures/pda1.eps}
\end{center}
\begin{itemize}
\item The finite control reads one symbol at a time from the input, and observes the symbol on top of the stack.
\item It bases its transitions on {\blue its state}, {\blue the input symbol}, and {\blue the symbol on top of the stack}.
\item On a transition, the PDA:
%\begin{minipage}{6cm}
\begin{enumerate}
\item Consumes an input symbol. 
\item Goes to a new state (or stays in the old). 
\item Replaces the top of the stack by a string
\end{enumerate}
\end{itemize}
\end{slide}

\begin{slide}{An informal example}
\begin{itemize}
\item Consider the grammar $P\ra aPb\vb \e$, and its language \[
L=\{a^nb^n:n\geq 0\}\] 

\item Suppose you are a simple machine whose only memory is a stack.

\item How would you recognize strings in this language?
\end{itemize}

\end{slide}

\begin{slide}
{An informal example}
Consider the grammar $P\ra aPb\vb \e$, and its language \[
L=\{a^nb^n:n\geq 0\}\] 
A PDA for $L$ has 3 states and operates as follows:
\begin{enumerate}
\item Keeps reading $a$s and pushing them onto the stack until it finds a $b$, which makes it pop an $a$ from the stack and transition to the next state.
\item Keeps reading $b$s and popping as many $a$s; if the ``start symbol''  is on top of the stack, then accepts.
\end{enumerate}
\end{slide}

\begin{slide}{PDA, formally}
A PDA is a 7-tuple $P=(Q, \Sigma, \Gamma, \delta, q_0, Z_0, F)$, where:
\begin{itemize}
\item $Q, \Sigma, q_0,$ and $F$, are our old friends;
\item $\Gamma$ is the stack alphabet;
\item $Z_0$ is the start symbol;
\item $\delta$ is the transition function %$\delta: Q\times\Sigma\times\Gamma \longrightarrow {\cal P}(Q\times \Gamma)$, e.g., 
$\delta(q, a, X)=\{(p,\gamma),\cdots\}$, where:
\begin{itemize}
\item $q$ is  a state, $a$ is an input symbol, $X$ is a stack symbol
\item $p$ is a state, and $\gamma$ is the string of stack symbols that {\blue replaces $X$} on top of the stack. E.g.:
\[
\begin{array}{l}
\delta(q, a ,X) =\{(p,\epsilon)\} \mbox{ Stack is popped.}\\
\delta(q, a, X)=\{(p, X)\} \mbox{ Stack is unchanged.}\\
\delta(q, a, X)=\{(p, YZ)\} \mbox{ $X$ replaced by $Z$; $Y$ pushed.}\\
\end{array}
\]
\end{itemize}
\end{itemize}
\end{slide}

\begin{slide}{A Graphical notation for PDAs}
In the notation used in our textbook, a transition diagram for a  PDA consists of the following elements:
\begin{itemize}
\item Like in finite automata, nodes represent states, and the start state and accept states are denoted as usual.
\item Arcs correspond to transitions of the PDA as follows:
\begin{center}
 \includegraphics[scale=.5]{figures/pdatrans.eps} means $(p, \alpha)  \in \delta(q, a, X)$
 \end{center}
\item Conventionally, $Z_0$ or $Z$ denote the start symbol for the stack.
\end{itemize}
Let's obtain a PDA for the language $\{a^nb^n, n\geq 0\}$ using JFLAP.
\end{slide}


\begin{slide}{A Nondeterministic PDA}
Consider the grammar $P\ra 0P0\vb1P1\vb \e$, and its language \[
L=\{ww^R:w\in\{0,1,\}^*\}\] 

A PDA for $L$ has three states and operates as follows:
\begin{enumerate}
\item Guesses that it is reading $w$. Stays in state $q_0$, and pushes the input symbol onto the stack. 
\item  Guesses that it is in the middle of $ww^R$, {\em i.e.}, $w$ would be on the stack. Goes spontaneously to state $q_1$. 
\item It is  now reading the head of $w^R$. Compares it to the top of the stack. If they match, pops the stack, and remains in state $q_1$. 
If they don't match, goes to sleep. 
\item If the stack is empty, goes to state $q_2$ and accepts. 
\end{enumerate}
\end{slide}



\begin{slide}{PDA for $L=\{ww^R:w\in\{0,1,\}^*\}$}
\begin{center}
\includegraphics[scale=.2]{figures/pda2.eps}
\end{center}

\[
P=(\{q_0, q_1, q_2\}, \{0,1\}, \{0, 1, Z_0\}, \delta, q_0, Z_0, \{q_2\}).
\] where $\delta$ is given by the table below
\begin{center}
\includegraphics[scale=.4]{figures/pda3.eps}
\end{center}
We assume the PDA accepts a string by consuming it and entering an accepting state.
\end{slide}

\begin{slide}{PDA: exercise}
Design a PDA that recognizes the language
\[
\{a^ib^jc^k:i,j,k\geq 1\mbox{ and }i=j\mbox{ or }i=k\}
\]
Provide a CFG for the language.
\end{slide}

\begin{slide}{PDA: exercise}
Design a PDA that recognizes the language
\[
\{w: w\; \mbox{contains as many 1s as 0s}\}
\]
Provide a CFG for the language.
\end{slide}

\begin{slide}{PDA: exercise}
Design a PDA that recognizes the language
\[
\{w: w\; \mbox{contains more 1s than 0s}\}
\]
Provide a CFG for the language.
\end{slide}


\begin{slide}{Instantaneous Descriptions (IDs)}
\begin{itemize}
\item $(q, w, \gamma)$ is an {\blue Instantaneous Description}  of a PDA  configuration.
\begin{itemize}
\item $q$ is the current state
\item $w$ is the remaining input
\item $\gamma$ is the contents the stack
\end{itemize}
\item Using IDs to represent a {\blue computation step by a PDA}: suppose  \includegraphics[scale=.5]{figures/pdatrans.eps}, then for all strings $w$, \[(q, aw, X\beta)\vdash (p, w, \alpha \beta)\]

%\[\mbox{Suppose }(p, \alpha)  \in \delta(q, a, X)\mbox{, then for } (q, aw, X\beta)\vdash (p, w, \alpha \beta)\] 
%\item \cvd is the  closure of $\vdash$, meaning ``zero or more computation steps (moves) of the PDA''.

\end{itemize}
\end{slide}

\begin{slide}{Computation as a sequence of IDs}
Example: Starting from the ID $(q, 010, Z)$, show all the reachable ID's for the PDA below.
\begin{center}
\includegraphics[scale=.7]{figures/pda2new.eps}
\end{center}
\end{slide}

\begin{comment}
\begin{slide}{Some properties of PDA}\label{sl:prop}
\begin{itemize}
\item If an ID sequence is a legal computation for a PDA, then so is the sequence obtained by adding an additional string to the second and/or third component. Formally: $\forall w \in\Sigma^*, \beta \in\Gamma^*$ :
\[
\mbox{if }(q, x, \mbox{\red \underline{$\alpha$}})\cvd(p, y, \mbox{\red \underline{$\beta$}}) \mbox{, then }(q, xw, \mbox{\red \underline{$\alpha\gamma$}} )\cvd(p, yw, \mbox{\red\underline{ $\beta\gamma$}})
\]
\item  If an ID sequence is a legal computation 
for a PDA, and some tail of the input is 
not consumed, then removing this tail from 
all ID's result in a legal computation sequence. 
\[
\mbox{if }(q, {\red\underline{ xw}}, \alpha)\cvd(p, {\red\underline{ yw}}, \beta)  \mbox{, then }(q, {\red \underline{x}}, \alpha )\cvd(p, {\red \underline{y}}, \beta)
\]
\end{itemize}
\end{slide}
\end{comment}



\begin{slide}{The languages of a PDA}
\begin{itemize}
\item For a DFA $D$, we saw that $L(D)=\{w: \hat{\delta}(q_0, w)\in F\}$, where $F$ is the set of accepting states.
\item For PDAs, there are two {\blue equivalent} approaches to define the languages they accept:
\begin{itemize}
\item Acceptance by final state
\item Acceptance by empty stack
\end{itemize}
%\item Notice, however, that for a given PDA $P$, the languages that $P$ accepts by final state and by empty stack are usually different.
\end{itemize}
\end{slide}
 
 \begin{slide}{Acceptance by final state}
\begin{itemize}
\item  Let \pda{P}\ be a PDA. 
\item The language accepted by $P$ by {\blue final state} is
 \[
 L_{f}(P ) = \{w : (q_0, w, Z_0)\cvd(q,{\red \e},\alpha), {\red q \in F} \}.
\]
Notice:
\begin{itemize}
\item $P$ consumes $w$ completely.
\item $P$ halts in an accepting state.
\item The stack might not be emptied.
\end{itemize}
\end{itemize}
\end{slide}


\begin{slide}{Acceptance by empty stack}
\begin{itemize}
\item Let \pda{P} be a PDA.
\item The language accepted by $P$ by {\blue empty stack} is
\[
L_{e}(P)=\{w:(q_0,w,Z_0)\cvd (q, {\red \e}, {\red \e})\}
\]
Notice:
\begin{itemize}
\item  $q$ can be any state
\item $P$ at the same time consumes $w$ and empties the stack
\item $F$ is redundant in the definition of the PDA and is usually left off in the representation.
\end{itemize}
\end{itemize}
\end{slide}

\begin{slide}{From Empty Stack to Final State}
{\bf Theorem}: If $P_N=(Q, \Sigma, \Gamma, \delta, q_0, Z_0)$ accepts by empty stack, then $\exists$ PDA $P_F$ that accepts by final state such that $L_e(P_N)=L_f(P_F)$ .
\begin{itemize}
\item The idea behind the proof of this theorem is to build a PDA $P_F$ that simulates $P_N$ and accepts if $P_N$ empties the stack.
\end{itemize}
\begin{minipage}{5cm}
\includegraphics[scale=.3]{figures/pda5.eps}
\end{minipage}
\begin{minipage}{6cm}
\begin{itemize}
\item $\delta_F(p_0,\epsilon,X_0)=\{(q_0, Z_0X_0)\}$
\item Keep all state transitions of $P_N$.
\item Add transitions $\delta_F(q, \epsilon, X_0)=\{(p_f, \epsilon)\}$ for every state $q\in Q$
\end{itemize}
\end{minipage}
\[
(p_0,w,X_0)\avd{{P_F}}(q_0,w,Z_0X_0)\cavd{{P_N}}(q,\e,X_0)\avd{{P_F}}(p_f,\e,\e)
\]
\end{slide}


\begin{slide}{From Final State to Empty Stack}
{\bf Theorem}: If \pda{P_F} accepts by final state, then $\exists$ PDA $P_N$ that accepts by empty stack such that $L_f(P_F)=L_e(P_N)$.
\begin{itemize}
\item The idea behind the proof of this theorem is to build a PDA $P_N$ that simulates $P_F$ and accepts if $P_F$ accepts by final state.
\end{itemize}
\begin{minipage}{5cm}
\includegraphics[scale=.3]{figures/pda8.eps}
\end{minipage}
\begin{minipage}{6cm}
\begin{itemize}
\item $\delta_N(p_0,\epsilon,X_0)=\{(q_0, Z_0X_0)\}$
\item Keep all state transitions of $P_F$.
\item Add transitions $\delta_N(q, \epsilon, X)=\{(p_f, \epsilon)\}$ for every state $q\in F$ and symbol $X\in\Gamma\cup\{X_0\}$.
\end{itemize}
\end{minipage}
\end{slide}

\begin{comment}
\begin{slide}{PDA conversion: example}
Consider the PDA $P$ below and its language $L_f(P)$. Convert $P$ to another PDA $P_1$ that accepts by empty stack such that $L_f(P)=L_e(P_1)$
\begin{center}
\includegraphics[scale=.7]{figures/pda2new.eps}
\end{center}
\end{slide}
\end{comment}

\begin{slide}{Equivalence of PDAs and CFGs}
\begin{itemize}
\item We shall see that PDAs and CFGs are equivalent in power: both represent the same class of languages.
\item We will do that by showing how to obtain a PDA from a CFG, and vice-versa.
\item Our objective is to prove the following theorem:
{\em \begin{quote}
A language is context-free {\large\blue iff} some PDA recognizes it.
\end{quote}}
Therefore, two lemmas:
\begin{itemize}
\item[{\blue\bf L1}:] If a language is context-free, then some PDA recognizes it.
\item[{\blue\bf L2}:] If a PDA recognizes some language, then it is context-free.
\end{itemize}
\end{itemize}
\end{slide}

\begin{slide}{Proving L1}
\begin{itemize}
\item Proof idea: show how to convert a CFG $G$ to a PDA $P$.
\item We will design $P$ to simulate leftmost derivations of strings according to $G$.
\item The stack will contain the strings of the derivations.
\end{itemize}
The derivation of the string $aaabbb$ according to the grammar below gives us a hint on the operation of the PDA.
\[
\begin{array}{l}
A\ra aAb \vb \e\\\ \\
A\dra aAb\dra aaAbb \dra aaaAbbb \dra aaabbb
\end{array}
\]
%Now, let's see on the board how a PDA would accept this string by empty stack. And as we do it, we shall design the PDA.
\end{slide}

\begin{comment}
\item The PDA's nondeterminism will come in handy, as it allows it to guess the sequence of correct substitutions.
\item $P$'s design sketch: $P$ will push on the stack left-sentential forms of $G$; and it will go through these sentential forms, making one substitution after another. Eventually it {\blue may} arrive at a configuration (a string) that contains only terminals. Then $P$ accepts if this string matches $w$.
\end{itemize}
\end{slide}
\end{comment}

\begin{slide}{Proof idea of L1 (cnt.)}
This is how a PDA would accept a string based on the grammar rules:
\begin{itemize}
\item Place the start variable on the stack.
\item Repeat the following steps forever:
\begin{enumerate}
\item[a)] If the top of the stack is a variable $A$, nondeterministically select a rule $A \ra \gamma$ and substitute $A$ on the stack by $\gamma$.
\item[b)] If the top of the stack is a terminal symbol $a$, read the next symbol from the input and compare it to $a$. If they match, pop the stack, repeat (b). If they don't match, reject this branch of the nondeterminism.
\item[c)] If the top of the stack is $Z_0$, pop it without reading a symbol from input.
%\item[c)] If the top of the stack is P's start symbol, enter the accept state. Doing so accepts the input if it has all been read.
\end{enumerate}
\end{itemize}
\end{slide}

\begin{slide}{Obtaining a PDA from a CFG}
\begin{enumerate}
\item For each variable $A$ in grammar $G$,
\[
\delta(q, \epsilon, A)=\{(q, \beta)\;|\;a\rightarrow \beta\mbox{ is a production of $G$}\}
\]
\item For each terminal $a$, $\delta(q, a, a)=\{(q, \epsilon)\}$.
\end{enumerate}
\begin{minipage}{6cm}
Example: 
\[
A\ra aAb | \epsilon
\]
\end{minipage}
\begin{minipage}{5cm}
\begin{center}
\includegraphics[scale=.5]{figures/cfg-PDA.eps}
\end{center}
\end{minipage}
\end{slide}

\begin{slide}{Proof idea of L2}

\begin{itemize}
\item Our task: given a PDA $P$,  design a CFG that generates the strings that $P$ accepts.
\item Fundamental event in a PDA's history: the popping of a symbol from the stack while consuming input. 
\item We design a grammar that will have variables named {\blue\bf $A_{pXq}$} that generate all strings $w$ such that $(p, w, X\alpha) \cvd (q, \e, \alpha), \forall p, q, X$.
\item But to facilitate this task, we must make sure $P$  has the following features:
\begin{itemize}
\item It accepts by empty stack.
\item Each transition either pushes a symbol onto the stack or pops one off the stack, but not both at the same time.
\end{itemize}
\end{itemize}
\end{slide}

%\begin{slide}{On how $P$ computes $x$}
%First and last moves of $P$ computing $x$:
%\item (stack is empty) First move is a {\em push}
%\item Last move is a {\em pop} (stack becomes empty)
%\end{itemize}
%Therefore, considering that the stack is empty at the beginning of $P$'s computation of $x$, there are two possibilities:
%\begin{itemize}
%\item stack is empty {\em only at the end} of the computation of $x$
%\[
%A_{pq}\ra a A_{rs} b
%\]
%\item  stack is empty {\em before the end} of $P$'s computation of $x$
%\[
%A_{pq} \ra A_{pr}A_{rq}
%\]
%\end{itemize}
%\end{slide}

\begin{slide}{Describing $G$'s rules.}
Given $P=(Q, \Sigma, \Gamma, \delta, q_0, Z_0)$, let's construct $G=(V, \Sigma, R, S)$:
\begin{enumerate}
\item $V$ consists of the start symbol $S$ and all symbols $A_{pXq}$, where  $p,q\in Q$  and $X\in\Gamma$.
\item The production rules in $R$ are:
\begin{enumerate}
\item For all states $p$, $G$ has the production $S\ra A_{q_0Z_0p}$. Therefore, these productions say: $S$ will generate all strings $w$ that cause $P$ to empty its stack.
\item Let $\delta(q, a, X)=\{(r, Y_1Y_2\cdots Y_k),\cdots\}$, where $a$ is either a symbol of $\Sigma$ or \e, and $k\geq 0$.
\begin{itemize}
\item  if $k= 0$, i.e.,  $\delta(q,a,X)=\{(r,\e),\cdots\}$, then $A_{qXr}\ra \epsilon$.
\item if $k>0$, then for all lists of states $r_1r_2\cdots r_k$, $G$ has the production $A_{qX\underline{r_k}}\ra a \underbrace{A_{rY_1\underline{r_1}}}_{\mbox{pops $Y_1$}}\underbrace{A_{\underline{r_1}Y_2\underline{r_2}}}_{\mbox{pops $Y_2$}}\cdots \underbrace{A_{\underline{r_{k-1}}Y_k\underline{r_k}}}_{\mbox{pops $Y_k$}}$
\end{itemize}
\end{enumerate}
\end{enumerate}
\end{slide}

%\begin{slide}{Proving that $G$ does generate $x$}
%We need to prove that $A_{qXp}$ generates $w$ {\bf iff} $(q, w, X)\cvd (p,\e, \e)$\\\ \\
%{\bf Claim 1:} If $A_{qXp}$ generates $w$, then $(q, w, X)\cvd (p,\e, \e)$\\\ \\
%Proof by induction on the number of steps in the derivation of $w$ from $A_{qXp}$
%\begin{itemize}
%\item {\bf Basis:} One step. $G$ has to include the rule $A_{qXp}\ra w$. And this production does existwhich makes $P $p$ move to $q$, pop $X$ and accept $w$ on empty stack. Since that is $G$ includes $A_{pp}\ra \e$, clearly the input \e\ takes $P$ from $p$ with empty stack to $q$ with empty   stack.\qed
%\end{itemize}
%\end{slide}

%\begin{slide}{Proof (cont.)}
%Proving: If $A_{pq}$ generates $x$, then $x$ can bring $P$ from $p$ with empty stack  to $q$ with empty stack.

% {\bf induction:} Assume true for derivations with $k$ derivations, prove true to $k+1$.

%Suppose $A_{pq}\overset{*}{\Rightarrow} x$ with $k+1$ derivations. The first step an either be $A_{pq} \dra aA_{rs}b$ or $A_{pq} \dra A_{pr}A_{rq}$
%\begin{itemize}
%\item $A_{pq} \dra aA_{rs}b$: consider $x=ayb$. Notice that $A_{rs}\overset{*}{\dra} y$ in $k$ steps and $G$ includes $A_{pq}\ra aA_{rs}b$. Also  $\delta(p, a, \e)$ contains $(r,t)$, and $\delta(s, b, t)$ contains $(q,\e)$. Therefore, by IH and inspection of of $P$, it is clear that $x$ can bring $P$ from $p$ with empty stack to $q$ with empty stack. \qed 
%\item $A_{pq} \dra A_{pr}A_{rq}$: consider $x=yz$. Notice that $A_{pr}\overset{*}{\dra}y$ and $A_{rq}\overset{*}{\dra}z$ has at most $k$ steps. Therefore, proven by IH and inspection of $P$. \qed
%\end{itemize}
%\end{slide}


%\begin{slide}{Proving the other claim}
%{\bf Claim 2:} If $x$ can bring $p$ with empty stack to $q$ with empty stack, $A_{pq}$ generates $x$.
%\begin{itemize}
%\item Proof by induction on the number of steps in the computation of $P$.
%\end{itemize}
%The above proof shows that PDAs recognize the class of context-free languages.

% It also allows us to establish a relationship between regular languages and context free languages

% Because every regular language  (RL) is recognized by a FA and every FA is also a PDA, we now know that every RL is also a CFL.

%\end{slide}

\begin{slide}{Example}
Obtain a CFG for the following PDA:
\begin{center}
\includegraphics[scale=.6]{figures/pda6.eps}
\end{center}
\end{slide}

\begin{slide}{Deterministic PDA}
\begin{itemize}
\item Recall that DFAs and NFAs are equivalent in language recognition power.
\item The same does not happen with PDAs.
\item NPDAs are more powerful then DPDAs.
\item DPDAs accept a class of languages that is between RLs and CFLs.
\item Let's  start by defining what is a DPDA. Then show that DPDA languages include all RLs.
\end{itemize}
\begin{center}
\includegraphics[scale=.4]{figures/pda9.eps}
\end{center}
\end{slide}

\begin{slide}{Deterministic PDA (DPDA)}
A PDA \pda{P} is deterministic {\blue iff}:
\begin{itemize}
\item $\delta (q,a,X)$ is always empty or a singleton.
\item If $\delta(q, a, X)$ is nonempty, then $\delta(q, \e, X)$ must
be empty. 
\end{itemize}
But before analyzing a DPDA, let's see the source of nondeterminism in the PDA for $\{ww^R: w\in \{0,1\}^*\}$
\begin{center}
\includegraphics[scale=.35]{figures/pda2.eps}
\end{center}
\end{slide}

\begin{slide}{Example of a DPDA}
Let us define the following language
\(
L_{wcwr} = \{wcw^R : w \in \{0, 1\}^*\}
\)

Then $L_{wcwr}$ is recognized by the following DPDA
\begin{center}
\includegraphics[scale=.3]{figures/dpda1.eps}
\end{center}
Unlike in the previous example, all move choices for this PDA are {\blue deterministic}.
\quiz{pda.1}
\end{slide}

\begin{comment}
\begin{slide}{RLs and deterministic PDA}%$L(DPDA)$: the language of DPDA}
L(DPDA) includes all RLs.
\begin{description}
\item[Theorem:] If $R$ is regular, then $R = L(P )$ for some DPDA $P$.
\item[Proof:] Since R is regular there is a DFA $A$ such that
$R = L(A)$. Let 
\[
A = ({\blue Q},{\blue \Sigma},\delta_A,{\blue q_0},{\blue F})
\]
 We define the DPDA
 \[
P = ({\blue Q},{\blue \Sigma},\{Z_0\},\delta_P,{\blue q_0},Z_0,{\blue F}),
\]
where $\delta_P (q, a, Z_0) = \{(\delta_A(q, a), Z_0)\}$,  for all $p,q\in Q$, and $a\in\Sigma$.

With a simple induction on $|w|$ one can show
\[(q ,w,Z )\cavd{P }(p,\e,Z )\mbox{\bf\blue\ iff\ }\hat{\delta_A}(q ,w)=p
\]
\end{description}
\end{slide}

\begin{slide}{$L(DPDA)$: the language of DPDA (cnt.)}
\begin{itemize}
\item We have seen that $RL\subseteq  L(DPDA)$
\item Also, it is easy to see that  $L_{wcw^R}\in L(DPDA)-RL$. Hint: we use the PL to show that $L_{wcw^R}$ is not regular.
\item There are CFL that are not in L(DPDA), e.g., $L_{ww^R}$.
\item The proof that $L_{ww^R} \not\in L(DPDA)$ is complex, but its intuition is transparent.
\begin{itemize}
\item Suppose $P$ is a DPDA that accepts $L_{ww^R}$ and $w=0^n110^n0^n110^n$.
\item To recognize $w$, $P$ would have to use the stack to count the 0's, emptying its stack to recognize the second sequence of 0's.
\item if  it sees next an identical string, it must accept.
\item However, since its stack is empty, it cannot remember what $n$ was.
\end{itemize}
\end{itemize}
\end{slide}
\end{comment}

\begin{slide}{Exercise}
Provide a DPDA for the following languages:
\[
L=\{0^n1^m: n\leq m\}
\]
\[
L=\{0^n1^m:n\geq m\}
\]
\end{slide}

\end{document}

begin{comment}
\begin{slide}{Proving $x\in L(P) \Leftarrow x=ww^R$}
\begin{center}
\includegraphics[scale=.4]{figures/pda2.eps}
\end{center}
\begin{itemize}
\item We assume $x=ww^R$. Then, we can see (by inspecting the PDA) that the following is a legal computation sequence.
\[
\begin{array}{l}
(q_0, ww^R, Z_0) \cvd(q_0, w^R, w^RZ_0)\vdash(q_1, w^R, w^RZ_0)\cvd\\(q1,\e, Z_0)\vdash(q_2,\e, Z_0)
\end{array}
\]
\end{itemize}
\end{slide}
 
\begin{slide}{Proving $x\in L(P) \dra x=ww^R$}
\begin{center}
\includegraphics[scale=.4]{figures/pda2.eps}
\end{center}
\begin{itemize}
\item Observe that the only way the PDA can enter 
$q_2$ is if it is in state $q_1$ with an empty stack. 
\item Then, we'll show by induction in $|x|$ that 
\[
(q_0, x, \alpha)\cvd(q_1,\e, \alpha) \dra x = ww^R
\]
{\bf Basis:} If $x=\e$ then $x$ is a palindrome. \qed
 \end{itemize}
 \end{slide}
 
\begin{slide}{Proving {\small $(q_0, x, \alpha)\cvd(q_1,\e, \alpha) \dra x = ww^R$}}

{\bf Induction:} Suppose $x = a_1a_2\cdots a_n$, $n > 0$, and the IH holds for shorter strings. There are two moves for the PDA from the ID $(q_0,x, \alpha)$
\begin{itemize}
\item The spontaneous $(q_0, x, \alpha)\vdash(q_1, x, \alpha)$.  Now $(q1, x, \alpha)\cvd(q_1,\e, \beta)$ implies that $\beta< \alpha$, which implies $\beta\neq \alpha$. 
\item Loop and push. A computation ending in $(q_1, \e, \alpha)$ would look like this:
\(
(q_0, a_1a_2\cdots a_n, \alpha)\vdash(q_0, a_2\cdots a_n, a_1\alpha)\vdash\cdots\vdash
(q_1, a_n, a_1\alpha)\vdash(q_1,\e, \alpha)
\). 

Thus, $a_1=a_n$ and $(q_0, a_2 \cdots a_n, a_1\alpha)\cvd(q_1, a_n, a_1\alpha)$.

From theorem in Slide \ref{sl:prop}, we can remove $a_n$. Therefore, $(q_0, a_2\cdots a_{n-1}, a_1\alpha)\cvd(q_1, \e, a_1\alpha)$.
Then, by IH $a_2\cdots a_{n-1}=yy^R$. Then $x=a_1yy^Ra_n$ is a palindrome.\qed
\end{itemize}
\end{slide}
\end{comment}

