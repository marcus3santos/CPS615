\documentclass{prosper}%[pdf,azure,slideColor,colorBG]{prosper}
\hypersetup{pdfpagemode=FullScreen}

\usepackage{graphicx}
\usepackage{graphics}
\usepackage{rotating}
\usepackage{amsmath}
\usepackage{epsfig}
\usepackage{verbatim}

\newcommand{\enfa}{{$\epsilon$-NFA}}
\newcommand{\e} {{\mbox{$\epsilon$}}}
\newcommand{\ra}{\mbox{$\;\rightarrow\;$}}
\newcommand{\vb}{\mbox{$\;|\:$}}
\newcommand{\dra}{\mbox{$\;\Rightarrow\;$}}
\newcommand{\cdra}{\mbox{$\;\stackrel{*}{\Rightarrow}\;$}}
\newcommand{\lmd}{\mbox{$\;\underset{lm}{\Rightarrow}\;$}}
\newcommand{\lmdstar}{\mbox{$\;\underset{lm}{\stackrel{*}{\Rightarrow}}\;$}}
\newcommand{\rmdstar}{\mbox{$\;\underset{rm}{\stackrel{*}{\Rightarrow}}\;$}}

\newcommand{\rmd}{\mbox{$\;\underset{rm}{\Rightarrow}\;$}}\newcommand{\cfg}[1]{$#1 = (V, \Sigma, R, S)$}
\newcommand{\qed}{{$\;\;\Box$}}

\begin{document}
\begin{slide}{Parse Trees}

\begin{itemize}
\item If $w\in L(G)$, for some CFG, then $w$ has a {\blue parse tree} representing the {\blue syntactical structure} of $w$.
\begin{minipage}{4cm}
\[
\begin{array}{l}
A \ra 0A1\vb B\\
B\ra \#
\end{array}
\]
\end{minipage}
\begin{minipage}{5cm}
\includegraphics[scale=.25]{figures/cfl1.eps}
\end{minipage}
%\[
%A\dra0A1\dra00A11\dra000A111\dra000B111\dra000\#111
%\]

\item Like derivations, a parse tree is an alternative representation for verifying if a string is in the language defined by a CFG.
\item The process used for constructing a parse tree for a given string is similar to a derivation of the string. 
%\item If a grammar is ambiguous, there could be many parse trees for the same string.
%\item Ideally, we want languages to be unambiguous. But we cannot always remove the ambiguity.
\end{itemize}
\end{slide}

\begin{comment}
\begin{slide}{Constructing Parse Trees}
Let \cfg{G} be a CFG. A tree is a parse tree for $G$ if: 
\begin{itemize}
\item Each interior node is labelled by a variable in $V$. 
\item Each leaf is labelled by a symbol in $V\cup \Sigma\cup\{\e\}$. 
Any \e-labelled leaf is the only child of its parent. 
\item If an interior node is lablelled $A$, and its children (from left to right) are labelled 
\[
X_1, X_2,\cdots , X_k 
\]
then $A \ra X_1X_2\cdots  X_k \in R$
\end{itemize}
\begin{minipage}{6cm}
Example: given this grammar, construct two parse trees.
\end{minipage}
\begin{minipage}{4cm}
\[
\begin{array}{l}
A \ra 0A1\vb B\\
B\ra \epsilon \vb bB
\end{array}
\]
\end{minipage}

\end{slide}
\end{comment}

\begin{slide}{Constructing a parse tree}
We are particularly interested in parse trees where: the {\blue yield} ({string of leafs, from left to right}) is a terminal string, and the root is the start symbol.
\begin{minipage}{6cm}
\[
\begin{array}{l}
E \ra I \vb E+E\vb E*E\vb (E)\\
I\ra a\vb b \vb Ia \vb Ib \vb I0 \vb I1
\end{array}
\]
\end{minipage}
\begin{minipage}{4cm}
\includegraphics[scale=.33]{figures/cfl6.eps}
\end{minipage}
%\begin{minipage}{4.5cm}

The {\blue yield} of the tree is $a*(a+b00)$.
%\end{minipage}
%\begin{minipage}{4cm}
%\includegraphics[scale=.33]{figures/cfl2.eps}
%\end{minipage}
\end{slide}

\begin{slide}{Inference, Derivations, and Parse Trees}
\begin{itemize}
\item Let \cfg{G}\ be a CFG, and $A\in V$. 
\item The following methods to determine if $w$ is in the language of $A$ are {\blue equivalent}: 
\begin{itemize}
%\item Recursive inference
\item Derivations: $A\cdra w$, $A\lmdstar w$, or $A\rmdstar w$
\item Construction of the parse tree with root $A$ and yield $w$.
\end{itemize}
\end{itemize}
\end{slide}

\begin{slide}{Exercise}
\begin{itemize}
\item  Provide a grammar for the language $0^*1^n0^n1^*, n>1$.
\begin{comment}
\[
\begin{array}{l}
S\ra A1B\\
A \ra 0A \vb \e\\
B \ra 0B \vb 1B \vb \e
\end{array}
\]
\end{comment}
\item Provide a parse tree for the following string:
\[
0110011
\]
\end{itemize}
\end{slide}

\begin{slide}{Ambiguity in Grammars and Languages}
\begin{itemize}
\item A grammar captures the structure of a string through the  parse tree.
\item But is this structure always unique?
\item Depending on the application, uniqueness of the structure is highly desirable, e.g., compilers and translators.
\item A grammar is ambiguous if a  string in the language has two different leftmost (or rightmost) derivations or parse trees.
\end{itemize}
Let's see an example.
\end{slide}

\begin{slide}{Ambiguity in Grammars}
\begin{itemize}
\item A grammar \cfg{G} is ambiguous if there is a string in $\Sigma^*$ that has more than one parse tree. 
\item Example: In the grammar:
\(
E\ra I \vb E+E \vb E*E \vb (E)\cdots
\)
The sentential form {\blue\bf $E+E*E$} has two different {parse trees}:
\[
\begin{array}{l}
E\dra E+E \dra E+E*E\\
E\dra E*E \dra E+E*E
\end{array}
\]
\begin{center}
\includegraphics[scale=.3]{figures/clf7.eps}
\end{center}
\end{itemize}
\end{slide}

\begin{slide}{Removing ambiguity from grammars}
\begin{itemize}
\item Good news: there are ad hoc methods to reduce and remove ambiguity
\item Bad news: there is no general algorithm to remove umbiguity. Worse yet: some grammars are inherently ambiguous.
\end{itemize}
Let's study some techniques for spotting and reducing ambiguity in grammars.
\end{slide}

\begin{slide}{Techniques for reducing ambiguity}
We'll consider three grammar structures that often lead to ambiguity:
\begin{itemize}
\item $\epsilon$ rules like $S\rightarrow \epsilon$
\item Symmetric recursive rules like $S\rightarrow SS$
\item Rules that lead to ambiguous attachment of optional postfixes, e.g., $S\rightarrow aS | aSb$
\end{itemize}
Why $\epsilon$ rules are problematic? Consider $S\rightarrow SS | a|\epsilon$. 
\begin{itemize}
\item There are many possible derivations/parse trees for the string $a$ as we can use $S\ra SS$ repeatedly, and then get rid of the unnecessary $S$s by using $S\ra \epsilon$. E.g.:
\end{itemize}
\[
\begin{array}{l}
S\dra SS\dra SSS \dra aSS\dra aS \dra a\\
S\dra SS\dra S\dra a
\end{array}
\]
\end{slide}

\begin{slide}{Eliminating \e-rules}
\begin{itemize}
%\item  \e-productions, although convenient, are not essential.
\item {\em  Basic idea}: Suppose $A$ is {\blue nullable} (i.e., $A \cdra \e$). We'll then replace a rule like
like \(
C \ra BAD
\)
with
\(
 C\ra BAD, C\ra BD
\)
 and delete any rules with body \e.
 \end{itemize}
 Algorithm {$RemoveEps(G)$}, where $G=(V,T, R, S)$:
 \begin{enumerate}
 \item {\em Obtain the set of all nullable symbols, $n(G)$,  in $G$}:
 \begin{itemize}
 \item {\bf Basis}: For all rules $A\ra \e\in R$, include $A$ in $n(G)$. 
 \item {\bf Induction}: For all rules $A	\ra C_1C_2\cdots C_k \in R$.\\ If $\{C_1,C_2,\cdots, C_k\}	\subseteq	n(G)$, then include $A$ in $n(G)$.
 \end{itemize}
 \item {\em Obtain the new grammar $G_1$}:  for each rule $A\ra X_1X_2\cdots X_k$ of $R$, suppose $m$ of the $k$ $X_i$'s s are nullable. Then $G_1$ will contain $2^m$ versions of this rule, where the nullable $X_i$'s in all combinations are present or absent.
\end{enumerate}
\end{slide}

\begin{slide}{Eliminating \e-rules: example}
\begin{itemize}
\item Let $G$ be $S\ra AB, A\ra aAA\vb\e, B\ra bBB\vb\e$
\item Now $n(G) = \{A, B, S\}$. The first rule will become: $S \ra AB\vb A\vb B$,
the second $A \ra aAA\vb aA\vb aA\vb a$, and the third $B \ra bBB\vb bB\vb bB\vb b$
\item We then delete the redundant rules, and end up with grammar $G_1$ :
\(
S \ra AB\vb A\vb B, A \ra aAA\vb aA\vb a, B \ra bBB\vb bB\vb b
\)
\end{itemize}
Ok, I got it. But what if L(G) contains  $\epsilon$ and it is important to retain it? E.g.:
\[
\begin{array}{l}
S\rightarrow (S)\\
S \rightarrow SS\\
S\rightarrow \epsilon
\end{array}
\]
\end{slide}

\begin{slide}{A highly ambiguous grammar}
\[
\begin{array}{l}
S\rightarrow (S)\\
S \rightarrow SS\\
S\rightarrow \epsilon
\end{array}
\]
Parse tree for the string $(())()$
\begin{center}
\includegraphics[scale=.4]{figures/highlyambiguous.eps}
\end{center}
\end{slide}

\begin{slide}{When $\epsilon \in L(G)$}
We use the following algorithm to rewrite the grammar.\\\ 

{
$atMostOneEps(G)$:
\begin{enumerate}
\item $G'' = RemoveEps(G)$.
\item If start symbol $S$ of $G$ is nullable then
\begin{enumerate}
\item Create in $G''$ a new start symbol $S'$.
\item Add to R the two rules:
\[
\begin{array}{l}
	       		S' \rightarrow \epsilon\\  
				S' \rightarrow S
\end{array}
\]
\item Return $G''$.
\end{enumerate}
\end{enumerate}}
\end{slide}

\begin{slide}{Applying $atMostOneEps(G)$}
\begin{minipage}{6cm}
{\bf Original Grammar}:
\[
\begin{array}{l}
S\rightarrow (S)\\
S \rightarrow SS\\
S\rightarrow \epsilon
\end{array}
\]
\end{minipage}
\begin{minipage}{4cm}
{\bf  $removeEps$:}
\[
\begin{array}{l}
S\rightarrow (S)\\
S\ra ()\\
S \rightarrow SS\\
\end{array}
\]
\end{minipage}

\begin{minipage}{6cm}
{\bf Result of $atMostOneEps$:}
\[
\begin{array}{l}
S'\rightarrow \epsilon\\
S'\rightarrow S\\
S\rightarrow (S)\\
S\ra ()\\
S \rightarrow SS\\
\end{array}
\]

\end{minipage}
\end{slide}

\begin{slide}{But there is still ambiguity}
\begin{minipage}{5cm}
\[
\begin{array}{l}
S'\rightarrow \epsilon\\
S'\rightarrow S\\
S\rightarrow (S)\\
S\ra ()\\
S \rightarrow SS\\
\end{array}
\]
\end{minipage}
\begin{minipage}{6cm}
What about $()()()$?
\end{minipage}
\begin{center}
\includegraphics[scale=.3]{figures/stillambiguous.eps}
\end{center}
\end{slide}

\begin{slide}{Eliminating symmetric recursive rules}
\begin{minipage}{3cm}
\[
\begin{array}{l}
S'\rightarrow \epsilon\\
S'\rightarrow S\\
S\rightarrow (S)\\
S\ra ()\\

S \rightarrow SS\\
\end{array}
\]
\end{minipage}
\begin{minipage}{8cm}
\begin{itemize}
\item Replace    $S \rightarrow  SS$ with one of:
\[
\begin{array}{l}
    S \ra SS_1 \;\mbox{		 force branching to the left}\\
    S \ra S_1S 	\;\mbox{	 force branching to the right}
\end{array}
\]
\item add $S\ra S_1$ to the grammar, and 
\item change $S\ra (S), S\ra ()$ to $S_1\ra (S), S_1\ra ()$
\end{itemize}
\end{minipage}

\vspace{0.5cm}
So we get
\[
\begin{array}{l}
S'\rightarrow \epsilon\;\;\;\;\;\;S\ra SS_1\\
S'\rightarrow S\;\;\;\;\;\;S\ra S_1\\
S_1\rightarrow (S)\\
S_1\rightarrow ()
\end{array}
\]
\end{slide}


\begin{slide}{Ambiguous attachment}
A third source for ambiguity arises when constructs with optional fragments are nested. E.g.
\[
S \rightarrow aS \;|\; aSb\;|\;\epsilon
\]
Two different parse trees for the string {\tt aab}

\begin{center}
\includegraphics[scale=.3]{figures/pt1.eps}
\hspace{1cm}
\includegraphics[scale=.3]{figures/pt2.eps}
\end{center}
Exercise: provide a unambiguous grammar that recognizes the same language.
\end{slide}


\begin{slide}{Ambiguous attachment (cont.)}
The dangling else problem:
\\\ \\
\begin{verbatim}
<stmt> ::= if <cond> then <stmt>
<stmt> ::= if <cond> then <stmt> else <stmt>
\end{verbatim}
\ \\\ \\
Consider:
\\\ \\
{\tt if $cond_1$ then if $cond_2$ then $st_1$ else $st_2$}
\\\ \\
Should the {\tt else} go with with the innermost {\tt if} or with the outermost {\tt if}?

\end{slide}

\begin{slide}{The dangling else problem}
\begin{verbatim}
<stmt> ::= if <cond> then <stmt>
<stmt> ::= if <cond> then <stmt> else <stmt>
\end{verbatim}
\ \\\ \\
{\tt if $cond_1$ then if $cond_2$ then $st_1$ else $st_2$}

\begin{center}
\includegraphics[scale=.5]{figures/danglingelse.eps}
\end{center}
\end{slide}

\begin{slide}{The Java Fix}
The grammar guarantees that, if a top-level {\tt if} has an {\tt else}, then the embedded {\tt if} must also have one.
\begin{center}
\includegraphics[scale=.21]{figures/javafix.eps}
\end{center}
\end{slide}

\begin{slide}{Arithmetic Expressions: a better way}
Let's study this grammar:
\[
\begin{array}{l}
E \ra I \vb E + E \vb E * E \vb (E)\\
I \ra a \vb b \vb Ia \vb Ib \vb I0 \vb I1
\end{array}\]
\begin{itemize}
\item Problems: no precedence between * and +, and no left (or right) associativity.
\item How to solve that? Redesign the grammar so that parse trees would reflect such structure.
\end{itemize}
\end{slide}

\begin{slide}{Solution}
Introducing variables that represent "binding strength".
\begin{enumerate}
\item A {\blue factor} is an expression that cannot be broken apart by an adjacent $*$ or $+$. E.g.: Identifiers and a parenthesized expression. \hspace{1cm} $F\ra I\vb(E)$
\item A {\blue term} is an expression that cannot be broken by $+$. E.g.: $a * b$, or a {\blue factor}. \hspace{1cm}$T\ra F\vb T*F$
\item The rest are {\blue expressions}, i.e., they can be broken apart with $*$ or $+$. \hspace{1cm} $E\ra T\vb E+T$
\end{enumerate}
\begin{minipage}{6cm}
The original grammar:
\[
\begin{array}{l}
E \ra I \vb E + E \vb E * E \vb (E)\\
I \ra a \vb b \vb Ia \vb Ib \vb I0 \vb I1
\end{array}\]
\end{minipage}
\begin{minipage}{5cm}
The redesigned grammar:
\[
\begin{array}{l}
E \ra T \vb E+T\\
T \ra F \vb T*F\\
F \ra I \vb (E)\\
I \ra a \vb b \vb Ia \vb Ib \vb I0 \vb I1
\end{array}\]
\end{minipage}
\end{slide}

\begin{slide}{The redesigned grammar}
\[
\begin{array}{l}
E \ra T \vb E+T\\
T \ra F \vb T*F\\
F \ra I \vb (E)\\
I \ra a \vb b \vb Ia \vb Ib \vb I0 \vb I1
\end{array}\]
\begin{center}
Now the only parse tree for $a+a*a$ will be\\
\includegraphics[scale=.27]{figures/cfl9.eps}
\end{center}
\end{slide}


\begin{slide}{Exercise}
Suppose you are designing a programming language and want to specify the syntax for valid type declarations, e.g.:
\begin{verbatim}
int x, z=3;
real y;
complex s;
\end{verbatim}

Provide a grammar that defines the syntax for such type declarations.

Assumptions:
\begin{itemize}
\item variable identifiers: $x, y, z, s$
\item numbers: $0, 1, 2, \cdots, 9$
\item types: $int, real, complex$
\end{itemize}
\end{slide}
\end{document}
